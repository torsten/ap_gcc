%!TEX root = ../ap_gcc.tex

\section{Challenges} This section describes challenges we had to face while porting the CGA framework to Mac OS X and Linux.

\subsection{Complexity}

The first challenge was the complexity of CGA. The main application itself contains more than 70.000 lines of code. The whole distribution consists of about 170.000 lines of code.

Furthermore, analyzing an application using CGA requires several steps, like adjusting the build process of the application that is getting analyzed, patching the applications binary and post processing the data collected while running the application. Porting this tool chain to another operating system requires a deep understanding of all those processes on one hand, and on the other hand a good idea how to realize all those details on the target platforms.

\subsection{Platform specifics}

The whole CGA tool chain relies on lots of platform specific mechanisms such as executable file information extraction, patching and collecting of information from the dynamic linker. A big challenge was to find ways to get all the information needed on both target platforms since this is not covered by standardized POSIX APIs.

Parts of our implementation rely on GCC and GDB, which are both available on Linux and Mac OS X. They provided us with a good point of abstraction to hide platform specific details. But even with those tools, certain things behave differently on both platforms. Writing into a binary with GDB does not work on Mac OS X, but does work on Linux \todo{name reason}. Function call addresses reported by the GCC instrument function mechanism may be wrong on Linux. \todo{explain plt section and jump table walking} Just to name two examples.

DllMain is a mechanism which enables a developer of a dynamic library to react on events like getting loaded, getting unloaded, new threads attached and similar. This mechanism is not available on Mac OS X's .dylib system nor on Linux's .so system. \todo{add a footnote or so} So we had to make excessive use of lazy initialization, thread specific storage and similar constructs.

\subsection{Compiler specifics}

The main instrumentation mechanism is based on a feature provided by the GCC compiler.  The Microsoft Visual C++ compiler can insert calls to instrumentation functions right after a function was called and right before a function returns. The mechanism in general is the same using GCC, but the differences appear when it comes to details.

In the Visual C++ compiler, it is possible to compile a function \inlinecode{naked}, which removes functions prolog and epilog and lets the programmer implement them himself. This feature enables the function to have a certain view on the stack, because the implementation itself is responsible for creating the stack frame, adjusting stack pointers and so on. So as a \inlinecode{naked} function starts, it has the same view on the stack as the function which called it. This is great for the implementation of the instrumentation functions. GCC as well does provide this feature, but, it is not supported on x86 platforms. So we had to work around this situation.

Some data types, like \inlinecode{hash\_map} and \inlinecode{hash\_set}, which are not part of the C++ standard, have different names and reside in different namespaces on different compilers.  In GCC they also require the definition of custom helper functions to hash certain data types whereas the Visual Studio compiler provides more default implementations for this.

\subsection{IDE specifics}

While introducing the CMake based build system in CGA, we found ourselves in front of a complex and highly platform specific Visual Studio solution with lots of inter project dependencies. It contained lots of custom build steps, like Qt preprocessing steps (\toolname{uic} and \toolname{moc}) and post build steps to get, for example, unit testing data in place.

Furthermore the solution was a grown structure, so several obsolete code files still exist in the source tree, but were excluded from either the build process of the whole project.  Includes defined using the Visual Studio project settings were missing in the source files which actually needed them. Just to name a few pitfalls.

\subsection{Backport to Windows}

Porting CGA to Linux and Mac OS X was only the first step. After we applied all the changes to the source tree and integrated CMake as a build system, we had to backport to Windows and build CGA on Windows using the new CMake build system  and integrate the pre-compiled header support into this process.

\subsection{Merging}

Due to the requirement, that all the changes made during the seminar should be able to easily get applied to the trunk of CGA, we had to merge changes from the seminars main branch to our own one. Those changes again brought incompatibilities with GCC and even our unix fact extraction port which partly relies on code from the Windows fact extraction mechanisms.

When all the other project branches merged their changes into the seminars main branch, we as well pulled that changes into our branch to ensure Linux and Mac OS X compatibility of those projects.
