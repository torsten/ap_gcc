%!TEX root = ../ap_gcc.tex

\section{Results}

This section provides a short overview on what we achieved during the project.

\subsection{CMake build enviroment} We switched the build environment of CGA from Visual Studio to CMake. This enables the same build configuration to manage the build process on several platforms for several compilers by generating makefiles or project files  for the most common IDEs.

We tested this system for Visual C++ on Windows and for Makefiles on Linux and Mac OS X. It is very likely that this will also work without any tweaks for KDevelop, XCode and many many more.

\subsection{Callmon runtime library} We implemented a callmon runtime library for applications and libraries build with GCC. This library features a lock free path from call event to log file (even in multithreaded environments using atomic operations), starting and stopping of the logging process using file system event APIs and asynchronous I/O to handle writing of log files as efficient as possible. Finally we benchmarked I/O throughput of up to 70MB per second (MacBook Pro Unibody running ioQuake3).

\subsection{Patch and Patchclean} We implemented tools to patch executables and libraries on Linux and Mac OS X. These tools store patch information in a very efficient way to binary patch files, which consume very little disk space.

\todo{little disk space?}

\subsection{Metacreator} To port metacreator to Linux and Mac OS X we just implemented the DiaInterface. So we were able to port the whole tool with very little code changes.

We added a class which gets instantiated from the DiaFactory and then interfaces with GDB to obtain debugging information like file names, line numbers of functions and call sites. Caching allows to reduce the communication with GDB to a minimum.

\subsection{Cross platform fixes for the CGA Toolbar} Besides some fixes for cross platform compatibility, the CGA Toolbar is exactly the same and can be used as usual for starting and stopping the logging process.

\subsection{Cross platform fixes for the CGA application} Finally we fixed several issues in the CGA application to allow the operation on Windows and Unix platforms.
