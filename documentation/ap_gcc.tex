\documentclass[a4paper,11pt]{article}

\usepackage{pdfsync}

% Use utf-8 encoding for foreign characters
\usepackage[utf8]{inputenc} 
\usepackage[english]{babel} 
% \usepackage[T1]{fontenc} 

% Setup for fullpage use
\usepackage{fullpage}

% Uncomment some of the following if you use the features
%
% Running Headers and footers
%\usepackage{fancyhdr}
% Multipart figures
%\usepackage{subfigure}
% More symbols
%\usepackage{amsmath}
%\usepackage{amssymb}
%\usepackage{latexsym}
% Surround parts of graphics with box
\usepackage{boxedminipage}

% Package for including code in the document
\usepackage{listings}

% If you want to generate a toc for each chapter (use with book)
\usepackage{minitoc}

% url package
\usepackage[hyphens]{url}

% enable clickable
\usepackage[]{hyperref}
\hypersetup{pdfborder=0 0 0}


% This is now the recommended way for checking for PDFLaTeX:
\usepackage{ifpdf}

%\newif\ifpdf
%\ifx\pdfoutput\undefined
%\pdffalse % we are not running PDFLaTeX
%\else
%\pdfoutput=1 % we are running PDFLaTeX
%\pdftrue
%\fi
\ifpdf 
\usepackage[pdftex]{graphicx} \else 
\usepackage{graphicx} \fi

% command to highlight todos
\usepackage{color}
\definecolor{Orange}{rgb}{1,0.5,0}
\newcommand{\todo}[1]{\textsf{\textbf{\textcolor{Orange}{[[#1]]}}}}

% sets the indent of the paragraph
\setlength{\parindent}{0cm}

% our commands to mark-up stuff
\newcommand{\inlinecode}[1]{\emph{#1}}
\newcommand{\toolname}[1]{\texttt{#1}}
\newcommand{\pathname}[1]{\texttt{#1}}



\title{AP GCC - CGA tool chain on UNIX systems} 
\author{ Torsten Becker, Frederik Rudeck, Robert Timm }

\date{2009-08-01}

\begin{document}

% sets the langauge for the listings package	
\lstset{language=C++}

\ifpdf \DeclareGraphicsExtensions{.pdf, .jpg, .tif, .png} \else \DeclareGraphicsExtensions{.eps, .jpg} \fi

\maketitle
% introduction
The goal of software visualization is to give insight into complex, large-scale, existing software systems and their physical composition and relations. CGA - Call Graph Analyzer - is a software visualization infrastructure that is developed by the computer graphics systems chair of the Hasso-Plattner-Institute. The Call Graph Analyzer obtains information from various fact extraction tools and provides an interactive graph visualization system. CGA addresses the following different software engineering tasks:

\begin{itemize}
	\item Program understanding
	\item Debugging
	\item Performance analysis
	\item Quality Assurance
\end{itemize}

When CGA started, the intended operating system was Windows. The goal of this project was to port the whole CGA tool chain to Mac OS X and Linux systems.\\

The growing market share of Apple computers demands for cross platform tools which are not tied to the Windows platform. With our implementation, we provide an alternative fact extraction tool chain based on common unix development tools like GCC and GDB. This extends CGA to be used to instrument a wider range of projects, because now it is possible to analyze systems which support the GCC compiler.
\newpage

\tableofcontents
\newpage


% \addtolength{\parskip}{\baselineskip}
\setlength{\parskip}{0.27cm}


%!TEX root = ../ap_gcc.tex

\section{Results}

This section provides a short overview on what we achieved during the project.

\subsection{CMake build enviroment} We switched the build environment of CGA from Visual Studio to CMake. This enables the same build configuration to manage the build process on several platforms for several compilers by generating makefiles or project files  for the most common IDEs.

We tested this system for Visual C++ on Windows and for Makefiles on Linux and Mac OS X. It is very likely that this will also work without any tweaks for KDevelop, Xcode and many many more.

\subsection{Callmon runtime library} We implemented a callmon runtime library for applications and libraries build with GCC. This library features a lock free path from call event to log file (even in multithreaded environments using atomic operations), starting and stopping of the logging process using file system event APIs and asynchronous I/O to handle writing of log files as efficient as possible. Finally we benchmarked I/O throughput of up to 70MB per second (MacBook Pro Unibody running ioQuake3).

\subsection{Patch and Patchclean} We implemented tools to patch executables and libraries on Linux and Mac OS X. These tools store patch information in a very efficient way to binary patch files, which consume very little disk space.

\todo{little disk space?}

\subsection{Metacreator} To port metacreator to Linux and Mac OS X we just implemented the DiaInterface. So we were able to port the whole tool with very little code changes.

We added a class which gets instantiated from the DiaFactory and then interfaces with GDB to obtain debugging information like file names, line numbers of functions and call sites. Caching allows to reduce the communication with GDB to a minimum.

\subsection{Cross platform fixes for the CGA application and toolbar}

Finally we fixed several issues in the CGA application to allow the operation on Windows and Unix platforms.

Besides some fixes for cross platform compatibility, the CGA Toolbar is exactly the same and can be used as usual for starting and stopping the logging process.

\todo{genauer}
 

\newpage

\section{Challenges} This section describes challenges we had to face while porting the CGA framework to Mac OS X and Linux.

\subsection{Complexity} The first challenge was the complexity of CGA. The main application itself contains more than 70.000 lines of code. The whole distribution consists of about 170.000 lines of code. 

Furthermore, analyzing an application using CGA requires several distinct steps, like adjusting the build process of the application that is getting analyzed, patching the applications binary and post processing the data collected while running the application. Porting this tool chain to another operating system requieres a deep understanding of all those processes on one hand, and on the other hand a good idea how to realize all those details on the target platform.

\subsection{Platform specifics} The whole CGA tool chain relies on lots of platform specific mechanisms like binary patching and collecting of information from the dynamic linker. A big challenge was to find ways to get all the information needed on both target platforms. 

Parts of our implementation rely on GCC and GDB, which are both available on Linux and Mac OS X. They provided us with a good point of abstraction to hide platform specific details. But even with those tools, certain thing behave differently on both platforms. Writing into a binary with GDB does not work on Mac OS X, but does work on Linux. Function call addresses reported by the GCC instrument function mechanism may be wrong on Linux. Just to name two examples. 

TODO DLLMAIN

\subsection{Compiler specifics} The main instrumentation mechanism if based on a feature provided by the compiler. The Microsoft Visual C++ can insert calls to instrumentation functions right after a function was called and right before a function returns. The mechanism in general is the same using GCC, but the differences appear when it comes to details. 

On Visual C++, it is possible to compile a function \emph{naked}, which removes functions prolog and epilog and lets the programmer implement them himself. This feature enables the function to have a certain view on the stack, because the implementation itself is responsable for creating the stack frame, adjusting stack pointers and so on. So as a \emph{naked} function starts, it has the same view on the stack as the function which called it. This is great for the implementation of the instrumentation functions. GCC as well does provide this feature, but, it is not supported on x86 platforms. So we had to work around this situation.

Some data types, like hash\_map, which are not part of the C++ Standard, have different names and reside in different namespaces on different compilers.

\subsection{IDE specifics} While introducing the CMake based build system in CGA, we found ourselves in front of a complex and highly platform specific Visual Studio solution with lots of inter project dependencies. It contained lots of custom build steps, like Qt preprocessing steps (uic and moc) and post build steps to get, for example, unit testing data in place.

Furthermore the solution was a grown structure, so several obsolete code files still exist in the source tree, but are excluded from the build process. Includes defined using the Visual Studio project were missing in the source files which actually needed them. Just to name a few pitfalls.

\subsection{Backport to Windows}

\subsection{Mergen}

\subsection{Qt did a great job} In general, we have to say, that Qt did a great job. Without the platform independence of not only all the GUI code, it would not have been possible to port CGA in such a short time periode. 
 

\newpage

\section{Requirements} This section describes some version requirements for the tools we are using. 

\subsection{Summary}
\begin{itemize}
  \item A patched version of VRS (see below for details)
	\item GCC tested on 4.2, 4.3 (need minimum version 4.2)
	\item GDB tested on 6.3, 6.8
	\item BINUTILS tested on 2.19.1, XCode 3.1.3
\end{itemize}

\subsection{Patched version of VRS} In order to get CGA running on Mac OS X and Linux, we needed to apply three patches to VRS. One of them is necessary to get VRS to compile, the second one is a header only fix for VRS, without it, CGA would not compile and the third one is a workaround for a crash in VRS, happening when using CGA. The three patches can be downloaded at GitHub:

\begin{itemize}
\item http://github.com/torsten/ap\_gcc/blob/b4847070f55198fcee57102c959adc0c4c794b5c/vrs-trunk-r6691-gcc-build-fix.patch
\item http://github.com/torsten/ap\_gcc/blob/b4847070f55198fcee57102c959adc0c4c794b5c/vrs-trunk-r6691-linux-cga-context-foo-crash-workaround.patch
\item http://github.com/torsten/ap\_gcc/blob/b4847070f55198fcee57102c959adc0c4c794b5c/vrs-trunk-r6691-mac-qt4-debug-only.patch
\end{itemize}

Please note, that all the three patches are made for VRS trunk revision 6691.

\subsection{GCC Version} We need a GCC version, which is greater than or equal to version 4.2 because we are using some functionality which appeared in GCC version 4.2 for the first time.\\

For example with the function \emph{\_\_sync\_bool\_compare\_and\_swap} GCC provides us with a cross platform way to test and set a variable in an atomic way. This is used several times in the lock free callmon implementation and is therefor essential.\\

We as well tested our code on GCC 4.3 without any problems.

\subsection{GDB Version} Our implementation of the metacreator was tested on GDB version 6.3 (on Apple Mac OS X) and 6.8 (on Ubuntu Debian Linux).\\

We do not depend on any functionality which was introduced in version 6.3, so our implementation may also run on older and/or newer versions of GDB. But since we are using GDB in a terminal way (writing commands to it and reading its output), we highly depend on the formatting of GDBs output. Even between version 6.3 and 6.8 there were several small differences like additional line breaks in GDBs output.\\

But extending our implementation to handle more versions of GDB is as easy as fixing the regular expressions parsing the GDB output.

\subsection{Binutils} On Linux we are using \emph{objdump} and \emph{nm} from the binutils distribution. All our code was tested with version 2.19.1 of these tools. On Mac OS X we a using \emph{otool} and \emph{nm} as provided by Apple bundled with XCode version 3.1.3.\\

Like in the GDB case we are parsing the output using regular expressions, so changes in the output format of these tools are likely to break our parsing, but again, only some regular expression need to be adjusted.
 

\newpage

\section{CMake Build System}
keine special coding richtlinien

\subsection{Qt specific build steps}
wo landen die uic generierten dateien

\subsection{on Windows}

\todo{CMake GUI - grouped view}

\subsection{on Linux}

\subsection{on Mac OS X} 
 

\newpage

\input{chapters/UNIXfe} 

\newpage

%!TEX root = ../ap_gcc.tex

\section{Tutorial}
\label{sec:Tutorial}

This section shows how to profile a project step by step. In general, the process is quite similar to the process needed to profile a project with the Windows version of callmon. Big differences only appear in the build configuration due to differences in the compiler switches needed by GCC.

\subsection{Preparing the build process} How to prepare the build process.

\subsubsection{Parameters to GCC} This is a list of parameters needed by GCC when profiling with unix fact extraction.

\paragraph{Compile with -g} Enable debug information.

With this option enabled, GCC builds debug information into the resulting binary. This is needed by metacreator to resolve source file name, line numbers and other valuable debugging information.

Note that you can still compile with optimizations enabled (\inlinecode{-O2} and friends) since our mechanism finds multiple enters and exists per function.

\paragraph{Compile with -finstrument-functions} Enable instrumentation.

This GCC parameter is the key. With this option enabled, GCC inserts calls to instrumentation functions right after a function was called and right before a function returns. If you do not provide this option in the compilation process, no calls will be made to the unix callmon library, and therefor no profiling information can be retrieved.

\paragraph{Compile with -fno-inline} Disable inlining of functions.

This disables the inlining of functions which is done explicitly by the developer or automatically by the compiler to optimize execution speed by eliminating function call overhead.

Since we are profiling on a function execution level, we cannot profile inlined function, so all the functions inlined by the compiler cannot appear in the call graph. To be sure this cannot happen, use \inlinecode{-fno-inline} as a GCC option. You might skip this if you want. You still might get good profiling results for the calls you are interested in, but you have been warned!

\paragraph{Link unixcallmon library as the last library} Ensure the right profiling functions are used.

It might happen, that other libraries as well provide the profiling functions that unix callmon lib provides (glib is an example for that). If this happens, it is important to ensure that the versions of the unix callmon lib are the ones linked in. This is done by adding the \inlinecode{-lunixcallmon\_lib} parameter as the last one.

\paragraph{On Linux, link with -Wl,-Bsymbolic} Make function calls patchable.

\todo{explain this case more in detail, just happens with .so s, etc.}

Without this option, symbolic function call information is removed from the resulting binary on Linux. This prevents \toolname{cga\_patch} and \toolname{cga\_patchclean} from finding the call locations and makes it impossible to remove those calls from the binary. 

\paragraph{Build with absolute path to source} Ensure CGA can find source files.

To ensure that CGA can find the source files for the source code viewer, you need to provide GCC with the absolute path to the source file while compiling.

\subsubsection{Putting it all together}

If the project you want to analyze is build using make, you might want to add the following to the projects Makefile:

\begin{verbatim}
  CFLAG   += -g -finstrument-functions -fno-inline 
  LDFLAGS += -Wl,-Bsymbolic -L/path/to/callmonlib -lunixcallmon_lib
\end{verbatim}

\subsection{Building the application}

With all the above mentioned set up, you build your project as usual, e.g. by typing \toolname{make}.

To enable the following command lines to work you might have to move the executables \toolname{cga\_patchclean} and \toolname{cga\_patch} to some location in your \inlinecode{PATH} for symlink them accordingly.

\subsection{Patching the executable}

As soon as the build process has finished, you end up with an executable or library which has all the profiling mechanisms build in. At this point, every single function which was just compiled by GCC is now enriched by profiling logic. Since this may be a lot, you might want to exclude several functions or groups of functions from the profiling process. This is done by patching the binary. Technically, the calls to the instrumentation functions get overwritten by NOP operations, so almost no overhead is involved in calling functions removed from the profiling process.

\subsubsection{Patch clean} The first thing to do is call \toolname{cga\_patchclean} on the binary like this:
\begin{verbatim}
  $ cga_patchclean myBinary myBinary.patch
\end{verbatim}
The first parameter specifies the binary to patch. The second parameter specifies a patch file name. In this patch file, \toolname{cga\_patchclean} will write out all the locations from which profiling calls were removed along with the opcodes removed that realized the call. This information is needed by \toolname{cga\_patch} to re-include the call opcodes for certain functions in the next step.

\subsubsection{Patch} At this point, profiling logic is re-added to the functions of interest. This is done by specifying two groups of function name patterns in a pattern file. This is exactly the same like for Windows callmon.

Provide a list of patterns in the include section, as well a list of patterns in the exclude section. Then, call patch like this:

\begin{verbatim}
  $ cga_patch myBinary myBinary.patch myPatternsFile.txt
\end{verbatim}

You may leave the patterns file parameter empty to re-include all functions in the profiling process. Like this:

\begin{verbatim}
  $ cga_patch myBinary myBinary.patch
\end{verbatim}

\subsection{Using CGA Toolbar}

The CGA Toolbar can be used as usual. It needs to set up the environment variable \inlinecode{CALLMON\_HOME}. This variable has to contain the path, where the log files are created. So lets say, you have a directory structure like this:

\begin{verbatim}
  <working directory>/
  |
  |- myBinary
  |- myBinary.patch
  |- myPatternsFile.txt
\end{verbatim}

Create a directory where the CGA Toolbar can operate on, like this:

\begin{verbatim}
  $ mkdir -p logs/myBinary
\end{verbatim}

You end up with a directory structure like this:

\begin{verbatim}
  <working directory>/
  |
  |- logs/
  |  |
  |  |- myBinary/
  |     |
  |     |
  |
  |- myBinary
  |- myBinary.patch
  |- myPatternsFile.txt
\end{verbatim}

Now, fire up the CGA Toolbar like this:

\begin{verbatim}
  $ CALLMON_HOME="<working directory>" cgatoolbar
\end{verbatim}

The string \emph{myBinary} should now show up in the drop down menu in the CGA Toolbar. If you now hit the start button, CGA Toolbar will create a file called \pathname{callmon.cmd}:

\begin{verbatim}
  <working directory>/
  |
  |- logs/
  |  |
  |  |- myBinary/
  |     |
  |     |- callmon.cmd
  |
  |- myBinary
  |- myBinary.patch
  |- myPatternsFile.txt
\end{verbatim}

This tells callmon to log function calls. Hitting the stop button will remove this file. You are now ready to run your application and record profiling information.

\subsection{Running the application}

Run the application as usual for example like this:

\begin{verbatim}
  $ ./myBinary -someParameter=someValue
\end{verbatim}

If you now press start and stop on the CGA Toolbar, new traces will be generated. Each trace will reside in the \pathname{logs/myBinary/} directory. Each trace will be put into its own directory depending on the date and time the logging started at. So after you created several traces, you might end up with a directory structure like this:   

\begin{verbatim}
  <working directory>/
  |
  |- logs/
  |  |
  |  |- myBinary/
  |     |
  |     |- callmon.cmd
  |     |- 090229_143523
  |     |- 090229_143542
  |     |- 090229_143559
  |     |- 090229_143614
  |
  |- myBinary
  |- myBinary.patch
  |- myPatternsFile.txt
\end{verbatim}

Each trace directory now contains .cmlog files, each of them representing the events that occured in one thread and .modinfo files, that contain the dynamic library state when logging started and as well when logging ended. So the directory for one trace may look like this: 

\begin{verbatim}
  090229_143542/
  |
  |-profile_4243_b8bfa41d.cmlog
  |-profile_4243_b8bfd411.cmlog
  |-profile_4243_b1ad00d2.cmlog
  |-profile_4243_pre.modinfo
  |-profile_4243_post.modinfo
\end{verbatim}

The first number in the .cmlog filename describes the process identifier, the second number in hexadecimal describes the thread identifier. The .modinfo filenames as well contain the process identifier.

\subsection{Using Metacreator}

The next step is to enrich the collected information by running metacreator. Therefor you simple fire up metacreator and provide it with a traces directory as parameter, like this:

\begin{verbatim}
  $ metacreator ./logs/myBinary/090229_143542
\end{verbatim}

Depending on the amount of events you collected, metacreator will take some time now. For all the logged calls, metacreator will now resolve debugging information from the binary. Once finished, a new .callmon file was created in the traces directory. So it should look like this now:

\begin{verbatim}
  090229_143542/
  |
  |-profile_4243_b8bfa41d.cmlog
  |-profile_4243_b8bfd411.cmlog
  |-profile_4243_b1ad00d2.cmlog
  |-profile_4243_pre.modinfo
  |-profile_4243_post.modinfo
  |-profile_4243.callmon
\end{verbatim}

All the preparations are done now. You can now start up CGA and load the trace.

\subsection{Loading the trace(s) into CGA} Start the CGA executable, create a new project. Then, select manage traces, click the add button, select your .callmon file and let CGA import the trace.
 

\newpage

\section{Guideline - Code that builds on GCC and Visual C++}
This section contains a list of the most common problems we found in the code of CGA and their solution.

\subsection{Paths} To be valid on Windows and Unix platforms, a path \textbf{must not} contain backslashes as separators. The only valid path separator for both platforms is the slash symbol \textbf{/}. So a valid path looks like this: 
\begin{verbatim}
	"this/is/a/valid/path" 
\end{verbatim}

\subsection{Const correctness} 

\subsection{Templates} The GCC's parser for template type name behaves slightly different than the Visual C++ ones. For example this is a valid definition in Visual C++: 
\begin{verbatim}
	std::list<std::pair<int, int>> myListOfIntPairs; 
\end{verbatim}

This is \textbf{not} valid while compiling with GCC. You have to separate \textbf{>} symbols using a space, else, GCC will throw a parser error. So this is the valid equivalent, which compiles on GCC and Visual C++: 
\begin{verbatim}
	std::list<std::pair<int, int> > myListOfIntPairs; 
\end{verbatim}

\subsection{Member function declarations} When declaring a member function inside the class statement, some people tent to prepend the name of the class to the method name. This may increase readability when inheriting several levels:
\begin{verbatim}
    class A {
    public:
        virtual void A::funcFromA();
    };
    
    class B : public A {
    public:
        virtual void A::funcFromA();
        virtual void B::funcFromB();
    };
    
    class C : public B {
    public:
        virtual void A::funcFromA();
        virtual void B::funcFromB();
        virtual void C::funcFromC();
    };
\end{verbatim}

The problem is, this \textbf{is not} a valid syntax for GCC. You \textbf{must not} prepend the class name to the member function. So the above declaration is valid for GCC like this:
\begin{verbatim}
    class A {
    public:
        virtual void funcFromA();
    };
    
    class B : public A {
    public:
        virtual void funcFromA();
        virtual void funcFromB();
    };
    
    class C : public B {
    public:
        virtual void funcFromA();
        virtual void funcFromB();
        virtual void funcFromC();
    };
\end{verbatim}

\subsection{windows.h} You \textbf{must not} include windows.h because all the types and functions provided by windows.h are highly Windows specific and will not compile nor run on other platforms. In general you will find the same functionality in QtCore. When using QtCore's functionality, it is easy to compile and run the code on all the platforms supported by Qt.

\subsection{for each() vs. foreach() vs. for()} Visual C++ provides a construct which looks like this:
\begin{verbatim}
    for each(int i in myIntList) {
        // loop code here
    }
\end{verbatim}
This \textbf{is not} available on GCC. There this cannot compile on both compilers. But the for each way is handy, so a cross compiler alternative is again the usage of Qt. Qt provides a construct like this:
\begin{verbatim}
    foreach(int i, myIntList) {
        // loop code here
    }
\end{verbatim}
Using this construct the resulting code is again cross compiler compatible and stays readable and handy.

\subsection{stdext vs. \_\_gnu\_cxx vs. tr1} Datatypes like the hash\_map are currently not part of the C++ Standard Template Library. But compiler vendors provide extensions in their own namespaces. Visual C++ provides this in the stdext namespace, GCC up to version 4.2 in the \_\_gnu\_cxx namespace. Since version 4.3 of GCC, the hash\_map was moved to the namespace std::tr1 and renamed to unordered\_map. The new C++ standard C++0x is on it's way and will contain the unordered\_map. So it is very likely that a new namespace will contain unordered\_map. For now, we found the following solution to the problem:
\begin{verbatim}
    #if __GNUC__ == 4 && __GNUC_MINOR__ >= 2
    #  if __GNUC_MINOR__ == 2
    #    include <ext/hash_map>
    #    include <ext/hash_set>
    #    define HASHMAP_TYPE      __gnu_cxx::hash_map
    #    define HASHMULTIMAP_TYPE __gnu_cxx::hash_multimap
    #    define HASHSET_TYPE      __gnu_cxx::hash_set
    #    define HASHMAP_NAMESPACE_OPEN  namespace __gnu_cxx {
    #    define HASHMAP_NAMESPACE_CLOSE }
    #  else // __GNUC_MINOR__ > 2
    #    include <tr1/unordered_map>
    #    include <tr1/unordered_set>
    #    define HASHMAP_TYPE      std::tr1::unordered_map
    #    define HASHMULTIMAP_TYPE std::tr1::unordered_multimap
    #    define HASHSET_TYPE      std::tr1::unordered_set
    #    define HASHMAP_NAMESPACE_OPEN  namespace std { namespace tr1 {
    #    define HASHMAP_NAMESPACE_CLOSE }}
    #  endif
    #else // __GNUC__ != 4 && __GNUC_MINOR__ < 2
    #  error "unsupported gcc version, need gcc 4.2 or higher"
    #endif
\end{verbatim}
Yes, this is just the GCC part, to include Visual C++ too, it needs still a bit more code, which is as well included in our branch of CGA. So to keep the cross compiler compatibility the marcos from above should be used. 

\subsection{Qt is the key} Qt provides a great way to write platform independent code. QtCore contains lots of things which replace pthreads\_create() or WaitForSingleObject() which else would break cross platform compatibility. 

 

\newpage

%!TEX root = ../ap_gcc.tex

\section{Final words} So this is the end of the CGA porting project. Finally, we had to realize, that it was quite a challenge to port the fact extraction process to two new platforms. Mac OS X and Linux are quite similar, but when it comes to executables, dynamic libraries and other operating system specific internals, they differ a lot. We learned a lot about bloddy internals. This was hard work, but fun as well. 

Thanks to the CGA team for providing a cosy seminar atmosphere and always helping us when we had a question.

Torsten Becker

Frederik Rudeck

Robert Timm 
 

\bibliographystyle{plain} 
\bibliography{}

\end{document} 
